A digital twin is a virtual representation of a physical object that is designed to exactly reflect it. The
object under investigation — say, a wind turbine — is equipped with a variety of sensors that
monitor various aspects of its operation. These sensors collect data on the energy production,
temperature, weather conditions, and other characteristics of the physical object's performance.
This information is subsequently sent to a processing machine, where it is applied to a digital copy.
Once the virtual model has been given this information, it may be used to run simulations,
investigate performance concerns, and suggest improvements, all with the purpose of gaining
important insights that can later be applied to the original physical device. In this paper methods of
machine learning will be discussed, statistics, forecasting, mathematical equations, real-world
examples, and methodology for building digital twins.

 
